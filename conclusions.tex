\chapter{CONCLUSIONS AND OUTLOOK}
\label{conclusions}

We began this thesis by asking: what happens to an atom in an electric field? The work described here answered this question with unprecedented precision. We measured the static polarizability of potassium and rubidium with 0.5\% uncertainty, and measured polarizability ratios with 0.3\% precision. We measured a magic-zero wavelength of potassium -- the wavelength at which nothing happens to the atom in the electric field -- for the first time. We also developed a new atom beam velocity measurement technique, phase choppers, to enable even more precise polarizability measurements in the future. 

Looking forward, we can realistically anticipate improved measurements of static polarizabilities and magic-zero wavelengths in the Cronin lab. The experiments described in this thesis form the foundation of a long program of static and dynamic polarizability measurements in Arizona. At this time, new measurements of cesium, strontium, and ytterbium polarizabilities appear to be of the highest importance. We have already measured the polarizability of Cs with 0.1\% precision (Figure \ref{csPolNew}). We have also generated beams of Sr and Ba, as well as beams of metastable He and Ar. Preliminary results using a hall-of-mirrors interaction region have shown a 5-10 times increase in the precision of $\lambdaZero$ measurements.

Have we answered the question of what happens to an atom in an electric field well enough? What will we learn by continuing to improve upon polarizability and magic-zero wavelength measurements?

We believe that there is significant value in pursuing improved measurements of static and dynamic polarizability. Table \ref{polResultsRecValues} showed the state-of-the-art measurements and calculations of the static polarizabilities of the alkalis. Theory uncertainty is currently about a factor of 2-5 better than the experiment uncertainty for most of these atoms, but theory uncertainty is difficult to estimate and theorists still routinely request new benchmark measurements of polarizabilities. New experiments are underway to probe parity non-conservation in Cs and Yb, and these experiments will need higher precision measurements of polarizabilities to test the associated atomic structure calculations. Next-generation optical clocks also need higher precision measurements of polarizabilities to accurately correct for blackbody radiation shifts in the clock frequencies. 

Future measurements of magic-zero wavelengths will determine difficult-to-calculate matrix elements with high precision, and may provide a sensitive way to probe the core electron contribution to polarizabilities. The many-body physics involved in the calculation of core electron polarizabilities has wide applications in atomic, nuclear, and condensed matter physics.

Polarizability, magic-zero wavelength, and van der Waals potential measurements of small molecules will have applications ranging from testing density functional theory to nanotechnology. We are currently working on implementing an electron ionization detector and mass filter to enable these measurements with molecules and non-alkali atoms.

Finally, it is worthwhile to reflect on the fact that when the Bederson group started its polarizability measurement program in the early 1960s there were no proposals to measure parity non-conservation in atomic systems (although the idea had been considered and dismissed as impractical \cite{Der07}) nor had atomic clocks been built with such incredible precision that blackbody radiation became an important component in the error budget. However, the Bederson group recognized the wide applications of polarizability measurements and their utility as benchmarks for the rapidly growing field of atomic physics. Similarly, future atomic physics experiments may well benefit from our measurements of polarizabilities today.


